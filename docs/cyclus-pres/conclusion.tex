\begin{frame}
    \frametitle{Conclusion}
    Cyclus is a performant, expanding fuel cycle simulator that holds promise for future applications. It demonstrated its capability to:
    \begin{itemize}
        \item `Predict the past'
        \item Model transition scenarios
        \item Visualize important fuel cycle metrics
    \end{itemize}
\end{frame}

\begin{frame}
    \frametitle{Future Work Ongoing}
    \texttt{Cycamore} is adequate for rough analyses, but more accurate
    modules or additional tools would increase analysis fidelity
    \begin{itemize}
        \item Dynamic archetype parameters (e.g. \texttt{refuel\_time}
                changing in time or sampled from a distribution)
        \item In-module depletion (i.e. Using in-module SERPENT Reduced-order-model)
        \item Demand-driven deployment \footnotemark
        \item Database-based MSR simulator
    \end{itemize}
    \footnotetext{NEUP 16-10512}
\end{frame}


\begin{frame}
    \frametitle{Room for Improvement}
    To really take it to the next level, a more organized, supported
    effort is needed.
    \begin{itemize}
        \item Organized / Supported Code development and Quality Assurance (QA)
        \begin{itemize}
            \item Currently done by a Cylcus Developer Manager (25\% PostDoc)
            \item User support and bug fixing on Github done voluntarily
            \item Various archetypes are outdated
        \end{itemize}
        \item Easier User Interface (UI) for broader user base
        \begin{itemize}
            \item Input generation / validation / visualization (e.g. Fulcrum)
            \item Ouput visualization / postprocessing (e.g. ORION)
        \end{itemize}
        \item Cloud execution of Cyclus (Users `submit job')
        \begin{itemize}
            \item Avoid Installation issues for end-users
        \end{itemize}
    \end{itemize}
\end{frame}


\begin{frame}
    \frametitle{Why \Cyclus?}
    \Cyclus has a unique expandable nature due to its open-source-ness and
    modularity of models. Here are a list of interesting projects we came up with:
    \begin{itemize}
        \item Central database driven high fidelity fuel cycle simulation
        \begin{itemize}
            \item Every fuel cycle facility in the Evaluation and Screening report modeled
            \item Real-life designs (Reactor designs, Reprocessing technologies etc.)
            \item plug-and-play for user
        \end{itemize}
        \item GIS-based work (every agent can have a coordinate property)
        \begin{itemize}
            \item Transportation model
            \item Energy demand / region based deployment (i.e. EAGLE-I)
        \end{itemize}
        \item Various Metrics Connector
        \begin{itemize}
        	\item Postprocessor for Safeguard / Non-proliferation metrics
        \end{itemize}
    \end{itemize}
\end{frame}


\begin{frame}
	\frametitle{Synergy with ORNL}
	\begin{itemize}
		\item Software management and QA infrastructure
		\item Computational development team with GUI development
		\item Most importantly: the personnel resources
		\begin{itemize}
			\item Fuel Cycle simulation is a combination of all aspects of nuclear engineering
			\item Insight / data from Safeguards, Reactor Physics 
			\item \textbf{Provide questions that needs to be answered}
		\end{itemize}
	\end{itemize}
\end{frame}


\begin{frame}
	\frametitle{Synergy with ORNL}
	\begin{itemize}
		\item \Cyclus is a tool. It should be developed to fulfill the needs of its customers.
		\item ORNL is an awesome resource for those interesting questions.
	\end{itemize}
\end{frame}


\begin{frame}
	\frametitle{Conclusion II}
	\begin{itemize}
		\item \Cyclus, with partnership with ORNL, can become a one-stop tool for system integration analysis.
		\item To achieve this, a central, supported effort is needed for \Cyclus
		\item Staff can `plug in' their developed model (reactor design, fuel cycle facility etc) to see the model's performance in a larger system.
	\end{itemize}
\end{frame}




\begin{frame}
    \frametitle{Acknowledgements}
    Invaluable advice and help provided by:
    Joshua Peterson-Droogh, Kathryn Huff, Andrew Worrall, Eva Davison

    This summer sponsored by:
    NESLS program at ORNL
\end{frame}

\begin{frame}
    Thank you. Questions / Concerns / Ideas?
\end{frame}